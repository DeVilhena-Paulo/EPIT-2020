\documentclass[handout]{beamer}

\usepackage{graphicx}
\usepackage{tikz-cd}

\title{EPIT Lecture 5.3\\ Homotopy pushouts and descent}
\author{Egbert Rijke}
\date{Friday, April 16th 2020}

\setbeamertemplate{caption}{\raggedright\insertcaption\par}

\mathchardef\usc="2D
\newcommand{\N}{\mathbb{N}}
\newcommand{\Z}{\mathbb{Z}}
\newcommand{\UU}{\mathcal{U}}
\newcommand{\brck}[1]{\|#1\|}
\newcommand{\Brck}[1]{\left\|#1\right\|}
\newcommand{\trunc}[2]{\|#2\|_{#1}}
\newcommand{\Trunc}[2]{\left\|#2\right\|_{#1}}
\newcommand{\unit}{\mathbf{1}}
\newcommand{\sphere}[1]{S^{#1}}
\newcommand{\isnull}{\mathsf{is\usc{}null}}
\newcommand{\htpy}{\sim}
\newcommand{\apbinary}{\mathsf{ap\usc{}bin}}
\newcommand{\Glob}{\mathsf{Glob}}
\newcommand{\typeGlob}{\mathsf{type}}
\newcommand{\relGlob}{\mathsf{rel}}
\newcommand{\homGlob}{\mathsf{hom}}
\newcommand{\maphomGlob}{\mathsf{map}}
\newcommand{\cgrhomGlob}{\mathsf{cgr}}
\newcommand{\bihomGlob}{\mathsf{bihom}}
\newcommand{\mapbihomGlob}{\mathsf{map}}
\newcommand{\cgrbihomGlob}{\mathsf{cgr}}
\newcommand{\ct}{\bullet}
\newcommand{\isconstant}[2]{\mathsf{is\usc{}}(#1,#2)\mathsf{\usc{}constant}}
\newcommand{\ap}{\mathsf{ap}}
\newcommand{\interchange}{\mathsf{interchange}}
\newcommand{\refl}{\mathsf{refl}}
\newcommand{\eh}{\mathsf{eckmann\usc{}hilton}}
\newcommand{\blank}{\mathord{\hspace{1pt}\text{--}\hspace{1pt}}}
\newcommand{\EM}{\mathsf{EM}}
\newcommand{\baseS}{\mathsf{base}}
\newcommand{\loopS}{\mathsf{loop}}
\newcommand{\apd}{\mathsf{apd}}
\newcommand{\tr}{\mathsf{tr}}
\newcommand{\idfunc}{\mathsf{id}}
\newcommand{\mulcircle}{\mu}
\newcommand{\basemulcircle}{\mathsf{base\usc{}mul}_{\sphere{1}}}
\newcommand{\loopmulcircle}{\mathsf{loop\usc{}mul}_{\sphere{1}}}
\newcommand{\htpyidcircle}{H}
\newcommand{\basehtpyidcircle}{\alpha}
\newcommand{\loophtpyidcircle}{\beta}
\newcommand{\invcircle}{\mathsf{inv}_{\sphere{1}}}
\newcommand{\evbase}{\mathsf{ev\usc{}base}}
\newcommand{\eqhtpy}{\mathsf{eq\usc{}htpy}}
\newcommand{\apply}[2]{#1(#2)}
\newcommand{\equiveq}{\mathsf{equiv\usc{}eq}}
\newcommand{\succZ}{\mathsf{succ}}
\newcommand{\bool}{\mathsf{bool}}
\newcommand{\const}{\mathsf{const}}
\newcommand{\btrue}{\mathsf{true}}
\newcommand{\bfalse}{\mathsf{false}}
\newcommand{\ttt}{\mathsf{pt}}

\setbeamertemplate{navigation symbols}{}
\setbeamertemplate{footline}[frame number]{}

\begin{document}

\begin{frame}
  \maketitle
\end{frame}

\begin{frame}
  Many spaces in topology are constructed by attaching cells, i.e., they are defined as CW-complexes.
  \begin{example}
    We can define a sphere by gluing the boundary of a disk to a single point.
  \end{example}
  Such constructions satisfy universal properties. In type theory, we can define many of them as higher inductive type.\\[1em]

  In this lecture we will see basic constructions of types such as spheres and suspensions, and we will study their basic properties.
\end{frame}

\begin{frame}
  \frametitle{Homotopy pushouts}
  Given $A \stackrel{f}{\leftarrow} S \stackrel{g}{\rightarrow} B$, the pushout of $f$ and $g$ is a type $A\sqcup^S B$ that fits in a commuting square
  \begin{equation*}
    \begin{tikzcd}[ampersand replacement=\&]
      S \arrow[d,swap,"f"] \arrow[r,"g"] \& B \arrow[d,"j"] \\
      A \arrow[r,swap,"i"] \& A\sqcup^C B
    \end{tikzcd}
  \end{equation*}
  that satisfies the universal property of a pushout.
\end{frame}

\begin{frame}
  \begin{example}
    The circle fits in a pushout square
    \begin{equation*}
      \begin{tikzcd}[ampersand replacement=\&]
        \mathbf{2} \arrow[r] \arrow[d] \& \mathbf{1} \arrow[d] \\
        \mathbf{1} \arrow[r] \& \sphere{1}
      \end{tikzcd}
    \end{equation*}
    in whith the homotopy $H:\const_\baseS\circ\const_\ttt\htpy\const\baseS\circ\const_\ttt$ is given by
    \begin{align*}
      H(\btrue) & := \loopS \\
      H(\bfalse) & := \refl
    \end{align*}    
  \end{example}
\end{frame}

\begin{frame}
  \begin{example}
    The spheres can be defined recursively as pushouts:
    \begin{equation*}
      \begin{tikzcd}[ampersand replacement=\&]
        \sphere{n} \arrow[r] \arrow[d] \& \mathbf{1} \arrow[d] \\
        \mathbf{1} \arrow[r] \& \sphere{n+1}
      \end{tikzcd}
    \end{equation*}
  \end{example}
\end{frame}

\begin{frame}
  \frametitle{Homotopy pullbacks}
  Given $A\stackrel{f}{\rightarrow} X \stackrel{g}{\leftarrow} B$, the pullback of $f$ and $g$ is a type $A\times_X B$ that fits in a commuting square
  \begin{equation*}
    \begin{tikzcd}
      A\times_XB \arrow[d] \arrow[r] \& B \arrow[d,"g"] \\
      A \arrow[r,swap,"f"] & X,
    \end{tikzcd}
  \end{equation*}
  that satisfies the universal property of a pullback.
\end{frame}

\begin{frame}
  \begin{theorem}[Descent for pushouts]
    Consider a commuting cube
    \begin{equation*}
      \begin{tikzcd}[ampersand replacement=\&]
        \& S' \arrow[dl] \arrow[d] \arrow[dr] \\
        A' \arrow[d] \& S \arrow[dl] \arrow[dr] \& B' \arrow[dl,crossing over] \arrow[d] \\
        A \arrow[dr] \& X' \arrow[from=ul,crossing over] \arrow[d] \& B \arrow[dl] \\
        \& X
      \end{tikzcd}
    \end{equation*}
    In which the bottom square is a pushout square and the two vertical squares in the back are pullback squares. The following are equivalent:
    \begin{enumerate}
    \item The two vertical squares in the front are pullback squares.
    \item The top square is a pushout square.
    \end{enumerate}
  \end{theorem}
\end{frame}

\end{document}
